\documentclass{article}

\usepackage[latin1]{inputenc}
\usepackage[T1]{fontenc}
\usepackage[francais]{babel}

\usepackage{amsthm}
\usepackage{amsmath}
\usepackage{amssymb}
\usepackage{mathrsfs}
\usepackage{graphicx}

\title{Project Programmation 2 : Report part 1}
\author{Adrien \bsc{Bardes}, Enguerrand \bsc{Dezerces}}
\date{March 6, 2018}
                  
\begin{document}

\maketitle

\section{The game}
When the game starts, towns are generated at random positions and with a random number of citizen. The player is in charge of a train company and he can do the following actions :
\begin{itemize}
	\item Build a station in a town (there can not be more than one station per town)
	\item Build a rail between two towns having a station
	\item Build a train in a town having a station
\end
Each of these actions cost money
The player can also click on a town or a rail in order to get extra information about it
The player has access to the list of trains and can click on one of the elements of the list to get extra info about the train

\section{Architecture of the project}

The project is divided into 4 packages : 
\begin{itemize}
	\item engine : The core of the game where the logic is executed
	\item interface : All the classes used for the graphical user interface
	\item link : Utility classes in order to make the link between the engine and the interface
	\item utils : Utility classes that can be used everywhere in the project
\end

\subsection{Engine}
The towns graph is contained in the object \textit{World}, each node is an instance of the abstract class \textit{Town}, and each edge is an instance of the class \textit{Rail}.
The game progresses by tick, at each tick, the world state is updated and the changes are sent to the interface.
The world contains instance of the class \textit{Company}.
The rails and the trains belongs to the company. 
For now there is only one company, owned by the player.
When a town is updated, it generates a percentage of passengers which are sent to the neighbouring towns, according to the number of available trains.
The town contains an instance of the abstract class \textit{Station}. 
The station manages the available trains. 
When there is no train available for a specific destination, the station keeps in memory the number of waiting passengers for this destination. 
When a station tries to send passenger to a destination, it takes the first train available for this destination and it loads it with the passengers.
When a train is updated, it checks its destination and checks if it is arrived. 
If so, it calls the method unload of the class station, on itself. 
If not, it moves toward its destination.

\subsection{Interface}
The interface uses the \textit{ScalaFX} library.
It is divided in objects, each object manages a specific part of the interface.
\textit{GUI} is the main object, it is also the main class, indeed it extends \textit{JFXApp} (a scalaFX class) therefore it is launched first.
The other components are : 
\begin{itemize}
	\item ItemsButtonsBar
	\item AllTrainsInformationPanel
	\item GlobalInformationPanel
	\item LocalInformationPanel
	\item OneTrainInformationPanel
\end

\subsection{Link}

\subsection{Utils}

\section{Shortcomings}

\end{document}




